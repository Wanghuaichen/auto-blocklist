
\section{Introduction} 

Internet censorship is pervasive in China. Topics ranging from political
dissent and religious assembly to privacy-enhancing technologies are known to
be censored. Naturally, the list of sites that are filtered are not
public~\cite{fhouse:china}. When performing measurements of Internet
filtering, the inability to know what sites are blocked creates a circular
problem of discovering the sites to measure in the first place. To do so, various
third parties currently curate
lists of websites that are known to be censored, or ``block lists''. These
lists are used to both understand {what} content is censored in China and
how that censorship is implemented.

Instead of curating a block list by hand, Darer et al. proposed a
system called FilteredWeb that automatically discovers webpages that
are censored in China~\cite{darer2017filteredweb}. Their approach is
summarized in the following steps. First, keywords are extracted from
webpages on the Citizen Lab block list, a small, hand-curated
list. These are English words that are ranked through TF-IDF, a
technique which we describe in Section ~\ref{tf-idf}. Then, each
keyword is used as a query for a search engine, such as Bing. The
intuition is that censored webpages contain similar keywords. Finally,
each webpage that appears in the search results is tested for DNS
manipulation in China. These tests are performed by sending DNS
queries to IP addresses in China that don't belong to DNS servers. If
a DNS response is received, then, it is inferred that the request was
intercepted in China, which implies that the website is censored. Each
webpage that is censored is fed back to the beginning of the
system. FilteredWeb discovered 1,355 censored domains, 759 of which
are outside the Alexa Top 1,000.

In this paper, we build upon the approach of FilteredWeb in the
following ways. First, {\em we extract content-rich phrases
for search queries}. In contrast, FilteredWeb only uses single words
for search queries. These phrases provide greater context regarding
the subject of censored webpages, which enables us to find websites
that are very closely related to each other. For example, consider the
phrase \begin{CJK*}{UTF8}{gbsn}中国侵犯人权 \end{CJK*} (Chinese human
rights violation). When we perform the searches \texttt{Chinese},
\texttt{human}, \texttt{rights} , and \texttt{violation}
independently, we mainly get websites for Western media outlets, many
of which are known to be censored in China. By contrast, if we search
for \texttt{Chinese human rights violation} as a single phrase, then
we discover a significant number of websites related to Chinese
culture, such as homepages for activist groups in China and
Taiwan. Identifying and extracting such key phrases is a non-trivial
task, as we discuss later.

Second, we use natural language processing to parse Chinese text
when adding to the blocklist. In contrast, FilteredWeb only extracts English
words that appear on a webpage. As such, \textit{any website that is written
in simplified Chinese is ignored}, neglecting a significant portion of
censored sites. For example, there are many censored websites and blogs that
cover Chinese news and culture, and many of them only contain Chinese text. As
such, to discover region-specific, censored websites, such a system should be
able to parse Chinese text. Because Chinese is typically written without
spaces separating words, this requires the use of natural-language processing
tools.

Third, we make our block list public, in contrast to previous work. The
authors of FilteredWeb made their block list available to us for validation;
we have published our block list so others can build on
it~\cite{censorsearch-lists}.

In summary, we built and now maintain a large, public, culture-specific list
of websites that are censored in China. These websites cover topics such as
political dissent, historical events pertaining to the Chinese Communist
Party, Tibetan rights, religious freedom, and more. Furthermore, because many
of these website are written from the perspective of Chinese nationals and expatriates,
we are able to get first-hand accounts of Chinese culture that are
not present in other block lists. This new resource can help researchers who
are interested in studying Chinese censorship from the perspective of
marginalized groups that most affected by it.

In this paper, we make the following contributions:
\begin{itemize}
  \item We build upon the approach of FilteredWeb to discover censored
websites in China that specifically pertain to its culture. We do so
by extracting potentially sensitive Chinese phrases from censored
webpages and using them as search terms to find related websites.
  \item We build a list of 1125 censored domains in China, which is
12.5$\times$ larger than the standard list for censorship
measurements~\cite{citizenlab:block}. Furthermore, \textit{none of
these websites are on the largest block list
available}~\cite{darer2017filteredweb}.
  \item We perform a qualitative analysis of our block list to
    showcase its advantages over previous work.
\end{itemize}

The rest of the paper proceeds as follows. First, we describe our
approach to building a large, culture-specific block list for
China. This includes an in-depth analysis of the advantages of our
approach over previous work. Then, we describe three large-scale
evaluations that we performed. Each of these evaluations
produced qualitatively different results due to different
configurations of our system. Finally, we conclude
with a discussion of how the block list we built could be used by
researchers. We also briefly explore directions for future work.