\begin{abstract}
Internet censorship measurements rely on lists of websites to be
tested, or ``block lists'' that are curated by third parties.
Unfortunately, many of these lists are not public, and those that are
tend to focus on a small group of topics, leaving other types of sites
and services untested. To increase and diversify the set of sites on
existing block lists, we use natural language processing and search
engines to automatically discover a much wider range of websites that
are censored in China. Using these techniques, we create a list of 1125
websites outside the Alexa Top 1,000 that cover Chinese politics,
minority human rights organizations, oppressed
religions, and more. Importantly, \textit{none of the sites we discover are
present on the current largest block list}. The list that we develop
not only vastly expands the set of sites that current Internet
measurement tools can test, but it also deepens our understanding of
the nature of content that is censored in China. We have released both
this new block list and the code for generating it.
\end{abstract}